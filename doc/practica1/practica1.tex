\documentclass[12pt, a4paper]{report}
\usepackage[utf8]{inputenc}
\usepackage{afterpage}
\usepackage{listings}
\usepackage{graphicx}
\graphicspath{ {imagenes/} }
\begin{document}
\begin{LARGE}
\begin{center}

Instituto Politécnico Nacional

\bigskip
\bigskip

Escuela Superior de Cómputo

\bigskip
\bigskip

Gonzalez Mora Javier

\bigskip
\bigskip

Sistemas operativos

\bigskip
\bigskip

2CM7


\end{center}
\end{LARGE}

\newpage

\begin{center}

Indice

\end{center}

\bigskip
\bigskip

\begin{flushleft}
Explicacion teorica .............................................................3

\bigskip

Programas utilizados...........................................................4

\bigskip

Desarrollo de la practica......................................................5

\bigskip

Errores y soluciones.............................................................9

\bigskip

\end{flushleft}

\bigskip
\bigskip

\bigskip

\newpage




\begin{center}

Practica 1

\end{center}

\begin{flushleft}


El proceso de arranque o boot es la secuencia de pasos y eventos que ocurren al prender una computadora. Al conectar la energia la computadora se asegura mediante un "test" rapido que los elementos principales esten conectados y operando.


\bigskip

Se ejecuta el BIOS y el BIOS busca el cargador del sistema operativo o bootloader, la carga en la RAM y lo ejecuta.


\bigskip

Bootloader: El bootloader busca y carga el sistema operativo (en ocasiones reescribe parte del BIOS).

\bigskip
\bigskip

El bootloader se encuentra en los primeros 512 bytes de la memoria.
\end{flushleft}




\newpage

\begin{center}
Programas utilizados y como fueron utilizados
\end{center}

\bigskip
\bigskip

\begin{flushleft}
\begin{itemize}
\item NASM: Se uso para poder compilar el codigo escrito a lenguaje maquina.
\item QEMU: Se uso para poder virtualizar la imagen ISO previamente creada.
\item genisoimage: Se uso para poder generar la imagen ISO, la cual fue creada con el codigo en NASM.
\end{itemize}
\end{flushleft}

\newpage

\begin{center}
Desarrollo de la practica
\end{center}

\bigskip
\bigskip

\begin{flushleft}
Codigo del bootloader:

\bigskip
\bigskip

\lstinputlisting{bootloader.asm}
\end{flushleft}

\bigskip
\bigskip

\begin{center}
En la primera parte se uso NASM para compilar el codigo del archivo bootloader.asm que posteriormente se generaria un archivo bootloader.bin. Lo que hace el bootloader es rellenar de ceros hasta una longitud de 512 bytes, despues se coloca el MAGICK NUMBER en la posición 521 y 522.

\bigskip
\bigskip

Instruccion para compilar y generar el archivo.bin: nasm src/bootloader.asm -f bin -o  bin/bootloader.bin

Pantalla de resultados:

\bigskip
\bigskip

\includegraphics[scale=.3]{1.png}

\end{center}

\newpage

\begin{center}
Se crea un disco con el cargador:

Se crea un floppy disk:

Instrucciones: 
\bigskip

dd if=/dev/zero of=imagenes/escomos.flp bs=1024 count=1440

\bigskip

dd if=bin/bootloader.bin of=imagenes/escomos.flp seek=0 count=1 conv=notrunc

\bigskip

Pantalla de resultados:

\bigskip

\includegraphics[scale=.3]{3.png}

\bigskip

\includegraphics[scale=.3]{4.png}

\bigskip

Se corre el bootloader

\bigskip

Instrucciones: qemu-system-i386 -fda imagenes/escomos.flp
Pantalla de resultados:

\bigskip

\includegraphics[scale=.3]{5.png}

\bigskip

Virtualización:

\bigskip

\includegraphics[scale=.3]{6.png}


\end{center}

\newpage

\begin{center}
Problemas que se presentaron: 

El problema que se presento fue en la instruccion para generar el bootloader.bin ya que hacia falta una bandera, que en este caso era -o

\bigskip

Instruccion: nasm src/bootloader.asm -f bin -o  bin/bootloader.bin

\bigskip

Pantalla de resultados:

\bigskip

\includegraphics[scale=.3]{1.png}


\end{center}

\end{document}